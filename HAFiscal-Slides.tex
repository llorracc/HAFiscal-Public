
% 3. Different package requirements: Slides need different packages than the main document (e.g., siunitx instead of dcolumn)

% allow latex to find custom stuff
\input{@resources/tex-add-search-paths}  
\input{./_path-to-root.ltx} 

\documentclass[pdflatex,aspectratio=169]{beamer}

\usepackage{pdfsuppressruntime}
\usepackage{multirow}
\usepackage{siunitx}  % Better decimal alignment than dcolumn
\newbool{fullcon}\global\booltrue{fullcon}\boolfalse{fullcon} %full content, 30-40 min
\newbool{bundesb}\global\booltrue{bundesb}\boolfalse{bundesb} %reduced, 20 min presentation
\newbool{upenn}\global\booltrue{upenn}\boolfalse{upenn} %90 min presentation
\newbool{ntnu}\global\booltrue{ntnu}\boolfalse{ntnu} %60 min presentation
\newbool{ashoka}\global\booltrue{ashoka}\boolfalse{ashoka} %60 min presentation
\newbool{vblt}\global\booltrue{vblt}%\boolfalse{vblt} %60 min presentation

% _____________ Opening slide _______________________

\title[Stimulus]{Welfare and Spending Effects of Consumption Stimulus Policies}
\author{
  Christopher D.\ Carroll (JHU)
  \and
  Edmund Crawley (FED)
  \and
  William Du (JHU)
  \and
  Ivan Frankovic (BBK)
  \and
  H{\aa}kon Tretvoll (SSB)
}

\ifbool{vblt}{\date[\today]{4th Annual LAEF Conference \\ Vanderbilt University \\ \medskip 2025-04-04  \\ \medskip \medskip \medskip \href{https://econ-ark.org/}{\small Powered By} \\ \includegraphics[width=0.5in]{./@resources/econ-ark/econ-ark-logo-small.png}}}

\ifbool{ashoka}{\date[\today]{Ashoka University \\ \medskip 2025-03-19  \\ \medskip \medskip \medskip \href{https://econ-ark.org/}{\small Powered By} \\ \includegraphics[width=0.5in]{./@resources/econ-ark/econ-ark-logo-small.png}}}

\ifbool{ntnu}{\date[\today]{Norwegian University of Science and Technology \\ \medskip 2025-01-23  \\ \medskip \medskip \medskip \href{https://econ-ark.org/}{\small Powered By} \\ \includegraphics[width=0.5in]{./@resources/econ-ark/econ-ark-logo-small.png}}}

\ifbool{upenn}{\date[\today]{University of Pennsylvania, 2024-11-06  \\ \medskip \medskip \medskip \href{https://econ-ark.org/}{\small Powered By} \\ \includegraphics[width=0.5in]{./@resources/econ-ark/econ-ark-logo-small.png}}}

\ifbool{bundesb}{
      \date[\today]{CEF - July 6, 2023  \\ \medskip \medskip \medskip
        \href{https://econ-ark.org/}{\small Powered By} \\ \includegraphics[width=0.5in]{econ-ark-logo-small.png}}}{}

\ifbool{fullcon}{
    \date[\today]{SSB Fiscal Policy Workshop - May 25, 2023  \\ \medskip \medskip \medskip 
          \href{https://econ-ark.org/}{\small Powered By} \\ \includegraphics[width=0.5in]{econ-ark-logo-small.png}}}{}

\newcommand{\RNum}[1]{\uppercase\expandafter{\romannumeral #1\relax}}

\AtBeginSection[]{
  \begin{frame}
    \vfill
    \centering
    \begin{beamercolorbox}[sep=8pt,center,shadow=true,rounded=true]{title}
      \usebeamerfont{title}\insertsectionhead\par%
    \end{beamercolorbox}
    \vfill
  \end{frame}
}

\usepackage[font=small,skip=0pt]{caption}
\usepackage{booktabs}

\providecommand{\PermGroFac}{\Gamma}

\usepackage{econark-shortcuts}
\begin{document}
\bibliographystyle{econark}

\begin{frame}[plain]
  \titlepage

  \footnotesize{Viewpoints and conclusions stated in this paper are the responsibility of the authors alone
    and do not necessarily reflect the viewpoints of The Federal Reserve Board or The Deutsche Bundesbank.}
\end{frame}

% _____________ 1st section  ____________

\begin{frame}
  \frametitle{Motivation}
  \begin{itemize}[<+->]
    \itemsep = .5\bigskipamount
    \item
          Fiscal policies that aim to boost consumption spending in recessions have been tried in many countries in recent decades
    \item
          A lot of variation in such policies --- may be due to little guidance from traditional macroeconomic models on which policies most effectively\ldots
          \begin{itemize}
            \itemsep = .25\bigskipamount
            \item
                  increase output (a `GDP metric')
            \item
                  reduce misery (a `welfare metric')
          \end{itemize}
    \item
          Development of heterogeneous agent (HA) models shows that when heterogeneity (in e.g.\ wealth, income and/or education) is taken into account, the impact of income shocks depends on \textit{intertemporal marginal propensity to consume} or iMPC
    \item
          In addition, availability of rich micro data (e.g.\ in Norway) provide first credible measures of the iMPC
    \item
          \textbf{This paper}: Aim to evaluate three consumption stimulus policies in a HA model consistent with data on liquid wealth and \textit{intertemporal} MPCs
  \end{itemize}
\end{frame}

\begin{frame}
  \frametitle{Related literature}
  \small
  \begin{itemize}[<+->]
    \item
          \textbf{Effects of transitory income shocks}:
          Parker, Souleles, Johnson and McClelland (2013); Broda and Parker (2014); Fagereng, Holm and Natvik (2021); Ganong, Greig, Noel, Sullivan and Vavra (2022)
    \item
          \textbf{HA models consistent with high MPCs}:
          Kaplan and Violante (2014); Auclert, Rognlie and Straub (2018); Carroll, Crawley, Slacalek and White (2020); Kaplan and Violante (2022)
    \item
          \textbf{State dependent multipliers (ZLB)}:
          Christiano, Eichenbaum and Rebelo (2011); Eggertson (2011); Ramey and Zubairy (2018); Hagedorn, Manovskii and Mitman (2019)
    \item
          \textbf{Extended unemployment insurance}:
          Ganong, Greig, Noel, Sullivan and Vavra (2022); Kekre (2022)
    \item
          \textbf{Welfare measures in HA models}:
          Bhandari, Evans, Golosov and Sargent (2021); D{\'a}vila and Schaab (2022)
    \item
          \textbf{High MPCs and impatience}: Parker (2017)
  \end{itemize}
  \normalsize
\end{frame}

\begin{frame}
  \frametitle{Quantitative Economics}
  \begin{itemize}[<+->]
    \itemsep = .75\bigskipamount
    \item
          These are \textit{quantitative} questions: require \textit{quantitative} realism ...
    \item
          ... about the differences that make a difference
          \begin{itemize}[<+->]
            \itemsep = .25\bigskipamount
            \item
                  UI benefits (normally last 2 quarter)
                  \begin{itemize}
                    \item
                          not forever as conveniently assumed in many models
                  \end{itemize}
                  %      \begin{itemize}[<+->]
                  %      \item Is not Calvo!
                  %        \item Makes a big difference quantitatively
                  %      \end{itemize}
            \item
                  Distributions of income, wealth
                  \begin{itemize}
                    \item
                          Profoundly important for (i)MPCs
                  \end{itemize}
            \item
                  Differences in unemployment risks
            \item
                  Heterogeneity in income growth rates
          \end{itemize}
  \end{itemize}

  %\pause Treatment of Multiplier?

  \begin{itemize}[<+->]
    \item
          Interested in multipliers, but baseline is NOT a HANK model:
          \begin{itemize}[<+->]
            \itemsep = .25\bigskipamount
            \item
                  HANK mechanisms behind multipliers are complex
            \item
                  Away from ZLB, multipliers not necessarily much different in recessions
                  %\item Far from clear if timing is right
          \end{itemize}
  \end{itemize}

  \begin{itemize}[<+->]
    \itemsep = .25\bigskipamount
    \item
          Robustness Exercise: HANK model
  \end{itemize}
\end{frame}

\begin{frame}
  \frametitle{Quantitative Micro Realism}
  \begin{itemize}
    \itemsep = \bigskipamount
    \item
          Idiosyncratic income process: Friedman/Muth (transitory and permanent shocks)
          \providecommand{\permLvl}{}\renewcommand{\permLvl}{\permLvlInd}
          \begin{eqnarray*}
            \permLvlInd & - & \text{`permanent income'} \\
            \tranShkInd & - & \text{`transitory income shock'}  \\
            \permShk & - & \text{`permanent income shock'}
          \end{eqnarray*}
          \begin{equation*}
            \begin{gathered}
              \begin{aligned}
                \permLvlInd_{t+1} & = \PermGroFac^{e} \permLvlInd_{t} \permShk_{t+1} \\
                y_{t+1}           & = \permLvlInd_{t+1}\tranShkInd_{t+1}             \\
              \end{aligned}
            \end{gathered}
          \end{equation*}

    \item
          $\PermGroFac^{e}$: education-specific income growth
    \item
          Evidence for permanent shocks: See Crawley, Holm, and Tretvoll (2024)
  \end{itemize}
\end{frame}

%\begin{frame}\frametitle{Evidence?}
%  \providecommand{\var}{}\renewcommand{\var}{\mathrm{var}}
%  For $k>3$,
%  \begin{equation}
%    \var(\log y_{t+k}/y_{t}) = 2 \sigma^{2}_{\log \tranShkInd} + k \sigma^{2}_{\log \permShkInd}
%  \end{equation}
%  Millions of datapoints from Norwegian National Registry:
%  \begin{center}
%    \includegraphics[width=0.5\linewidth]{./Figures/norway_income_change_variance.png}
%
%    Source: SSB (Elin Halvorsen)
%  \end{center}
%  Also see Crawley, Holm, and Tretvoll (2024)
%\end{frame}

\begin{frame}
  \frametitle{Preferences, Beliefs, and Wealth}
  Infinite horizon model: target wealth depends on `Growth Impatience' condition:
  \begin{equation}
    \underbrace{
      \left(
      \frac{(\Rfree \; \DiscFac^{e,i})^{1/\gamma}}
      {\PermGroFac^{e} \; \Ex[\permShk^{-1}]}
      \right)
    }_{\text{'Growth Patience Factor'}}
    < 1
  \end{equation}

  \pause
  \emph{Degree} of impatience (1-GPF) determines \emph{size} of target
  \begin{itemize}[<+->]
    \item
          If everybody has same GPF, then target wealth is identical
    \item
          Fact: Wealth much more unevenly distributed than permanent income \\[1ex]
          $\Rightarrow$ need heterogeneity in GPF
    \item
          (If GPF $\geq 1$, target is $\infty$)
  \end{itemize}

  \only<1>{\hypertarget{ConsistentWithMicroData}{}}

  \pause
  We use
  \begin{itemize}[<+->]
    \item
          \textit{Ex-ante} heterogeneity in discount factors $\DiscFac^{e,i}$
    \item
          $\PermGroFac^{e}$ or $\Rfree$ would do as well
  \end{itemize}

\end{frame}

\begin{frame}
  \frametitle{Consistency With Micro Evidence (1)}
  \begin{columns}
    \begin{column}{0.3\linewidth}
      Liquid Wealth from \href{https://www.federalreserve.gov/econres/scfindex.htm}{Survey of Consumer Finances (SCF)}
    \end{column}
    \begin{column}{0.7\linewidth}
      \includegraphics[width=.9\linewidth]{\FigDir/LorenzPoints}
    \end{column}
  \end{columns}
  \medskip
  \begin{itemize}
    \item
          Education groups: $e\in\{$`Dropout', `Highschool' and `College'$\}$
    \item
          Each group has distribution of discount factors $\beta_{e,i}$
  \end{itemize}
\end{frame}

\begin{frame}
  \frametitle{Consistency With Micro Evidence (2)}
  % \begin{columns}
  %		\begin{column}{0.4\textwidth}  	
  Intertemporal MPC from Fagereng, Holm, Natvik (2021)
  \begin{center}
    \includegraphics[width=.55\linewidth]{\PathToRoot/Code/HA-Models/Target_AggMPCX_LiquWealth/Figures/AggMPC_LotteryWin}
  \end{center}
  Modeling device: `Splurge' in consumption
  %		\end{column}
  %	\end{columns}
\end{frame}

\begin{frame}
  \frametitle{Splurge consumption}
  \begin{itemize}
    \itemsep = .75\bigskipamount
    \item
          Exogenous fraction of income directly consumed
    \item
          Model consistent with spending patterns over time after a transitory income shock
    \item
          Evidence: High liquid wealth hh also have high MPCs
          \begin{itemize}
            \itemsep = .25\bigskipamount
            \item
                  Kueng (2018); Crawley and Kuchler (2023); Graham and McDowall (2024)
          \end{itemize}
    \item
          Possible microfoundations:
          \begin{itemize}
            \itemsep = .25\bigskipamount
            \item
                  Spending on durables (Browning and Crossley, 2009; Laibson et al., 2022)
            \item
                  A form of present bias (Indarte et al., 2024, Maxted et al., 2024)
          \end{itemize}
    \item
          Robustness: Model w/o splurge consumption
  \end{itemize}
\end{frame}

\begin{frame}
  \frametitle{Evaluation of consumption stimulus policies in the US}
  \begin{itemize}[<+->]
    \itemsep = .5\bigskipamount
    \item
          Policies we consider:
          \begin{itemize}[<+->]
            \itemsep = .25\bigskipamount
            \item
                  Stimulus check for \$1200 (means-tested)
            \item
                  Extension of unemployment benefits from 6 months to 1 year
            \item
                  Payroll tax cut by 2\% for 2 years
          \end{itemize}
          \medskip
    \item
          Motivation:
          \begin{itemize}[<+->]
            \itemsep = .25\bigskipamount
            \item
                  Economic Stimulus Act of 2008 (stimulus checks)
            \item
                  Tax Relief, Unemployment Insurance Reauthorization, and Job Creation Act of 2010 (UI extension and tax cut)
          \end{itemize}
          % \item Key features of the policies: 
          %   \begin{itemize}[<+->]
          %     \itemsep = .25\bigskipamount 
          %   \item Targeting 
          %   \item Timing of spending (overlap with recession!)
          %   \item Scalability 
          %   \end{itemize}
          \medskip
    \item
          Evaluation criteria:
          \begin{itemize}[<+->]
            \itemsep = .25\bigskipamount
            \item
                  Spending multipliers
            \item
                  Welfare (only recession-related welfare impact)
          \end{itemize}
  \end{itemize}
\end{frame}

\begin{frame}
  \frametitle{Preview of results}
  \begin{itemize}[<+->]
    \itemsep = \bigskipamount
    \item
          Welfare measure: Extension of UI benefits is the clear winner
          \begin{itemize}[<+->]
            \itemsep = .25\bigskipamount
            \item
                  Targeted at individuals with high MPCs and high recession-related welfare losses
            \item
                  But: higher spending may continue after recession is over
          \end{itemize}
    \item
          Spending multiplier: Stimulus check has the highest multiplier
          \begin{itemize}[<+->]
            \itemsep = .25\bigskipamount
            \item
                  Not well targeted, but increases income immediately
                  % \item Also: easy to scale up
          \end{itemize}
    \item
          Tax cut
          \begin{itemize}[<+->]
            \itemsep = .25\bigskipamount
            \item
                  Poorly targeted and much spending likely to occur after end of recession
          \end{itemize}
    \item
          Robustness in a HANK and SAM model
          \begin{itemize}[<+->]
            \itemsep = .25\bigskipamount
            \item
                  Very similar pattern for cumulative multipliers
          \end{itemize}
  \end{itemize}
\end{frame}

\section{Model}

\begin{frame}
  \frametitle{Household problem}
  \begin{itemize}[<+->]
    \item
          Idiosyncratic, stochastic income process $\mathbf{y}_{i,t}$
    \item
          Estimated splurge factor $\varsigma$: $\mathbf{c}_{sp,i,t} = \varsigma \mathbf{y}_{i,t}$
          \pause
    \item
          Remaining consumption $c_{opt,i,t}$ is chosen to maximize utility
          \begin{equation}
            \begin{gathered}
              \begin{aligned}
                \sum_{t=0}^{\infty}\beta_{e,i}^t (1-D)^t \mathbb{E}_0 u(\mathbf{c}_{opt,i,t}).
              \end{aligned}
            \end{gathered}
          \end{equation}
          ($D$: end-of-life probability, $u$: CRRA utility function)
    \item
          Budget constraint, given existing market resources $\mathbf{m}_{i,t}$ and income state, and a no-borrowing constraint:
          \begin{equation}
            \begin{gathered}
              \begin{aligned}
                \mathbf{m}_{i,t+1} & = R \underbrace{(\mathbf{m}_{i,t} - \mathbf{c}_{sp,i,t} - \mathbf{c}_{opt,i,t})}_{\geq 0 \text{ (no-borrowing constraint)}} + \mathbf{y}_{i,t+1}
              \end{aligned}
            \end{gathered}
          \end{equation}
          ($R$: exogenous gross interest rate)
  \end{itemize}
\end{frame}

\begin{frame}
  \frametitle{ Income process}

  \begin{itemize}[<+->]

    \item
          Income subject to transitory, unempl. and permanent shocks
          \begin{equation}
            \begin{gathered}
              \begin{aligned}
                \mathbf{y}_{i,t} =   \begin{cases}
                                       \xi_{i,t}\mathbf{p}_{i,t}, & \text{if employed}                 \\
                                       0.7 \mathbf{p}_{i,t},      & \text{if unemployed for $\leq$ 2q} \\
                                       0.5 \mathbf{p}_{i,t},      & \text{if unemployed $\ge$ 2q}
                                     \end{cases}
              \end{aligned}
            \end{gathered}
          \end{equation}
          ($\xi_{i,t}$: trans.
          shock, $p$: perm.
          income)

    \item
          `Permanent income':  $\mathbf{p}_{i,t+1} = \underbrace{\psi_{i,t+1}}_{\text{perm.
                shock}} \underbrace{\Gamma_{e(i)}}_{\text{educ.-specific growth}}\mathbf{p}_{i,t}$

          \pause
          \bigskip
          %      \item Employment status is subject to a Markov process
          %        \begin{itemize}[<+->]
          %        \item Unemployment rate education-specific (doubles in recession)
          %        \item Expected length of unemployment: 1.5q  (4q in recession)
          %        \end{itemize}
          %        
          %      \item Recession is given by an MIT shock; end of recession as a Bernoulli process (avg.
          %length of 6q)

    \item
          Model is a simplified model of households (no heterogeneity in hh size)
    \item
          Replacement rates reflect some degree of hh incurance (Rothstein and Valetta, 2017)

  \end{itemize}

\end{frame}

\begin{frame}
  \frametitle{ Employment status and recessions}
  \begin{itemize}
    \itemsep = \bigskipamount
    \item
          Emplyoment status is subject to a Markov process
          \begin{itemize}
            \itemsep = .5\bigskipamount
            \item
                  Employed consumer: continue being employed or become unemployed
            \item
                  Unemployed consumers: receives benefits for two quarters
          \end{itemize}

    \item
          Bureau of Labor Statistics: Report unemployment rates by education group

    \item
          Recession is given by an MIT shock
          \begin{itemize}
            \itemsep = .5\bigskipamount
            \item
                  Unemployment rate doubles in each education group
            \item
                  Expected length of unemployment increases from 2 to 4q
            \item
                  End of recession occurs as a Bernoulli process calibrated for an avg.\ rec. length of 6q
          \end{itemize}
  \end{itemize}
\end{frame}

\begin{frame}
  \frametitle{Aggregate demand effects \\
    \small (as in Krueger, Mitman and Perri, 2016) \normalsize}
  \begin{itemize}[<+->]
    \itemsep = .5\bigskipamount
    \item
          Baseline: No feedback from aggregate consumption to income
    \item
          Extension: We allow for aggregate demand effects from consumption on income during the recession

    \item
          The AD effect is given by
          \begin{equation}
            \begin{gathered}
              \begin{aligned}
                AD(C_t) =   \begin{cases}
                              \Big(\frac{C_t}{\tilde{C}}\Big)^\kappa, & \text{if in a recession} \\
                              1,                                      & \text{otherwise} ,
                            \end{cases}
              \end{aligned}
            \end{gathered}
          \end{equation}
          where $\tilde{C}$ is the level of consumption in the steady state.

    \item
          Idiosyncratic income in the extension model is then given by
          \begin{equation}
            \begin{gathered}
              \begin{aligned}
                \mathbf{y}_{AD,i,t} = AD(C_t)\mathbf{y}_{i,t}.
              \end{aligned}
            \end{gathered}
          \end{equation}
  \end{itemize}
\end{frame}

%\begin{frame}
%	\frametitle{Three policies to fight the recession - Details}
%	
%	\begin{itemize}[<+->]
%		\item Stimulus check
%		\begin{itemize}[<+->]
%			\item Everyone receives a check for \$1,200 in q1 of the recession
%			\item Check is means-tested: Full check if perm. income $\leq$ \$100k; Falls linearly for higher incomes and zero for those $\geq$ \$150k
%		\end{itemize}
%		
%		\item Extended unemployment benefits
%		\begin{itemize}[<+->]
%			\item Unemployment benefits are extended from 2 to 4 q
%			\item Extension occurs regardless of whether recession ends
%		\end{itemize}
%		
%		\item Payroll tax cut
%		\begin{itemize}[<+->]
%			\item Employees payroll tax rate is reduced such that income rises by 2\% for 8q	
%		\end{itemize}
%	\end{itemize}
%	
%	Policies are debt-financed and repayed much later
%\end{frame}

\begin{frame}
  \frametitle{Parameters --- by education group \hyperlink{sli:paramsSame}{\beamerbutton{More parameters}} \hyperlink{sli:policies}{\beamerbutton{Policy parameters}}}
  \label{sli:paramsByEd}
  \hypertarget{Parameters}{}
  \begin{tabular}{lccc}
    \hline
    \multicolumn{4}{l}{Parameters calibrated for each education group}                                                \\
                                                                   & Dropout        & Highschool     & College        \\ \hline
    Percent of population                                          & \phantom{0}9.3 & 52.7           & 38.0           \\
    Avg.\ quarterly PI of ``newborn'' agent (\$1000)               & \phantom{0}6.2 & 11.1           & 14.5           \\
    Std.\ dev.\ of $\log($PI$)$ of ``newborn'' agent               & 0.32           & 0.42           & 0.53           \\
    Avg.\ quarterly gross growth rate of PI ($\Gamma_e$)           & 1.0036         & 1.0045         & 1.0049         \\
    Unemployment rate in normal times (percent)                    & \phantom{0}8.5 & \phantom{0}4.4 & \phantom{0}2.7 \\
    Probability of entering unemployment ($\pi_{eu}^{e}$, percent) & \phantom{0}6.2 & \phantom{0}3.1 & \phantom{0}1.8 \\
    Probability of leaving unemployment ($\pi_{ue}$)               & 0.667          & 0.667          & 0.667          \\ \hline
  \end{tabular}
  \begin{itemize}
    \item
          Mincer (1991) and Elsby and Hobjin (2010): Education groups differ in the incidence of unemployment, not its duration
  \end{itemize}
\end{frame}

\ifbool{fullcon}{

  \section{Parametrization}

  \begin{frame}
    \frametitle{Parametrization --- Strategy}
    \begin{itemize}[<+->]
      \itemsep = \bigskipamount
      \item
            Step 1: Estimate the splurge factor in a Norwegian version of the economy --- match iMPCs from FHN (2021)
      \item
            Step 2a: Calibrate a set of parameters that affect all education groups equally
      \item
            Step 2b: Calibrate a set of parameters that match features of the different education groups
      \item
            Step 3: Estimate a discount factor distribution for each education group to match within-group distribution of liquid wealth
            \begin{itemize}[<+->]
              \itemsep = .25\bigskipamount
              \item
                    $\beta_e$: center of discount factor distribution
              \item
                    $\nabla_e$: spread of discount factor distribution
              \item
                    Uniform distribution, approximated with 7 different types
            \end{itemize}
    \end{itemize}
  \end{frame}

  \begin{frame}
    \frametitle{Step 1: iMPC from FHN (2021)}
    \centering
    %	\includegraphics[width=3in]{\FigDir/AggMPC_LotteryWin}
    \includegraphics[width=3in]{\PathToRoot/Code/HA-Models/Target_AggMPCX_LiquWealth/Figures/AggMPC_LotteryWin}
    \begin{itemize}[<+->]
      \itemsep = .5\bigskipamount
      \item
            Estimated splurge factor: $\varsigma = 0.31$; MPC across wealth distrubtion and K/Y untargeted but close to targets
      \item
            Zero splurge ($\varsigma = 0$): cannot match iMPC, wealth-dep. MPCs and K/Y-ratio at the same time
    \end{itemize}
  \end{frame}

  \begin{frame}
    \frametitle{Parameters --- same for all types  \hyperlink{sli:policies}{\beamerbutton{Policy parameters}} }
    \hypertarget{Parameters}{}
    \begin{tabular}{lcS[table-format=1.3]}
      \hline
      \multicolumn{3}{l}{Parameters that apply to all types}                                                                             \\ \hline
      Parameter                                                   & Notation                     & {Value}                               \\ \hline
      Risk aversion                                               & $\gamma$                     & 2.0                                   \\
      Splurge                                                     & $\varsigma$                  & 0.249                                 \\
      Survival probability, quarterly                             & $1-D$                        & 0.994                                 \\
      Risk free interest rate, quarterly (gross)                  & $R$                          & 1.01                                  \\
      Standard deviation of transitory shock                      & $\sigma_\xi$                 & 0.346                                 \\
      Standard deviation of permanent shock                       & $\sigma_\psi$                & 0.0548                                \\
      Unemployment benefits replacement rate (share of PI)        & \textcolor{red}{$\rho_b$}    & \textcolor{red}{0}.\textcolor{red}{7} \\
      Unemployment income w/o benefits (share of PI)              & \textcolor{red}{$\rho_{nb}$} & \textcolor{red}{0}.\textcolor{red}{5} \\
      Avg. duration of unemp. benefits in normal times (quarters) &                              & 2                                     \\
      Avg. duration of unemp. spell in normal times (quarters)    &                              & 1.5                                   \\
      Probability of leaving unemployment                         & $\pi_{ue}$                   & 0.667                                 \\
      Consumption elasticity of aggregate demand effect           & $\kappa$                     & 0.3
      \\ \hline
    \end{tabular}
  \end{frame}

  \begin{frame}
    \frametitle{Step 2b: Parameters --- by education group}
    \label{sli:paramsByEd}
    \begin{tabular}{lccc}
      \hline
      \multicolumn{4}{l}{Parameters calibrated for each education group} 
                                                                     & Dropout        & Highschool     & College        \\ \hline
      Percent of population                                          & \phantom{0}9.3 & 52.7           & 38.0           \\
      Avg. quarterly PI of ``newborn'' agent (\$1000)                & \phantom{0}6.2 & 11.1           & 14.5           \\
      Std. dev.\ of $\log($PI$)$ of ``newborn'' agent                & 0.32           & 0.42           & 0.53           \\
      Avg. quarterly gross growth rate of PI ($\Gamma_e$)            & 1.0036         & 1.0045         & 1.0049         \\
      Unemployment rate in normal times (percent)                    & \phantom{0}8.5 & \phantom{0}4.4 & \phantom{0}2.7 \\
      Probability of entering unemployment ($\pi_{eu}^{e}$, percent) & \phantom{0}6.2 & \phantom{0}3.1 & \phantom{0}1.8
      \\ \hline
    \end{tabular}
  \end{frame}

  \begin{frame}
    \frametitle{Step 3: Estimation of discount factors}
    \begin{tabular}{lccc}
                                  & Dropout        & Highschool     & College        \\ \hline
      $(\beta_e, \nabla_e)$       & (0.719, 0.318) & (0.925, 0.077) & (0.983,0.014)  \\
      (Min, max) in approximation & (0.447, 0.991) & (0.859, 0.990) & (0.971, 0.995) \\
      \hline
    \end{tabular}
    \begin{tabular}{lccc}
      \multicolumn{4}{l}{ }                                                     \\ \hline
      \textbf{Estimation targets}              & Dropout & Highschool & College \\ \hline
      Median LW/ quarterly PI (data, percent)  & 4.64    & 30.2       & 112.8   \\
      Median LW/ quarterly PI (model, percent) & 4.64    & 30.2       & 112.8   %\\
      % $[20,40,60,80]$ pctiles of Lorenz curve (data) & $[0, 0.01, 0.6, 3.6]$ & $[0.06, 0.6, 3.0, 11.6]$ & $[0.2, 0.9, 3.3, 10.3]$ \\
      % $[20,40,60,80]$ pctiles of Lorenz curve (model) & $[0.0, 0.0, 0.5, 3.6]$ & $[0.04, 0.9, 3.7, 11.3]$ & $[0.3, 1.5, 4.0, \phantom{0}9.9]$
      \\ \hline
    \end{tabular}
    \begin{tabular}{lcccc}
      \multicolumn{5}{l}{ }                                                                \\ \hline
      \textbf{Non-targeted moments}          & Dropout & Highschool & College & Population \\ \hline
      Percent of total wealth (data)         & 0.8     & 17.9       & 81.2    & 100        \\
      Percent of total wealth (model)        & 1.2     & 16.8       & 82.0    & 100        \\
      Avg. annual MPC (model, incl. splurge) & 0.78    & 0.61       & 0.38    & 0.54
      \\ \hline
    \end{tabular}
  \end{frame}

  \begin{frame}
    \frametitle{Step 3: Visualization of match with SCF}
    \centering
    \includegraphics[width=4in]{\FigDir/LorenzPoints}
  \end{frame}

}{}

\section{Results}

\ifbool{bundesb}{

  \begin{frame}
    \frametitle{Impulse responses}

    \begin{columns}

      \begin{column}{0.5\textwidth}
        \small
        \begin{itemize}[<+->]
          \item
                Simulate policies in recessions lasting 1 to 20 q
          \item
                Construct probability-weighted sum across rec. lengths
        \end{itemize}
      \end{column}

      \begin{column}{0.4\textwidth}
        \footnotesize Stimulus check:
        \includegraphics[width=\linewidth]{\PathToRoot/Code/HA-Models/FromPandemicCode/Figures/recession_Check_relrecession}

      \end{column}

    \end{columns}

    \pause

    \begin{columns}

      \begin{column}{0.33\textwidth}
        \footnotesize Extension of UI benefits:
        \includegraphics[width=1.2\linewidth]{Code/HA-Models/FromPandemicCode/Figures/recession_UI_relrecession}
      \end{column}

      \begin{column}{0.33\textwidth}
        \footnotesize Payroll tax cut:
        \includegraphics[width=1.2\linewidth]{Code/HA-Models/FromPandemicCode/Figures/recession_taxcut_relrecession}
      \end{column}
    \end{columns}

  \end{frame}
}{}

\ifbool{fullcon}{

  \begin{frame}

    \frametitle{IRFs for stimulus check}

    \begin{columns}

      \begin{column}{0.33\textwidth}
        \begin{itemize}[<+->]
          \item
                Simulate check policy in recessions lasting  from 1 to 20 q
          \item
                Construct probability-weighted sum across rec. lengths
        \end{itemize}
      \end{column}

      \begin{column}{0.66\textwidth}
        \centering
        \includegraphics[width=\linewidth]{\PathToRoot/Code/HA-Models/FromPandemicCode/Figures/recession_Check_relrecession}
      \end{column}

    \end{columns}

  \end{frame}

  \begin{frame}
    \frametitle{IRfs for extension of unemployment benefits / payroll tax cut}

    \begin{columns}

      \begin{column}{0.50\textwidth}
        Extension of UI benefits:
        \includegraphics[width=1.2\linewidth]{Code/HA-Models/FromPandemicCode/Figures/recession_UI_relrecession}
      \end{column}

      \begin{column}{0.50\textwidth}
        Payroll tax cut:
        \includegraphics[width=1.2\linewidth]{Code/HA-Models/FromPandemicCode/Figures/recession_taxcut_relrecession}
      \end{column}
    \end{columns}

  \end{frame}

}{}

\begin{frame}
  \frametitle{Untargeted moments (1)}
  \begin{tabular}{lcccc}
    \multicolumn{5}{l}{Non-targeted moments by wealth quartile}                                                                                                        \\ \hline
                                                                  & WQ 4                   & WQ 3                   & WQ 2                   & WQ 1                    \\ \hline
    Percent of liquid wealth (data)                               & 0.14                   & 1.60                   & 8.51                   & 89.76                   \\
    Percent of liquid wealth (model, baseline)                    & 0.12                   & 0.98                   & 3.85                   & 95.0                    \\
    \textcolor{gray}{Percent of liquid wealth (model, Splurge=0)} & \textcolor{gray}{0.10} & \textcolor{gray}{1.07} & \textcolor{gray}{4.24} & \textcolor{gray}{94.60} \\
    \shortstack[l]{Avg. lottery-win-year MPC                                                                                                                           \\ (model, incl. splurge)} & 0.74 & 0.61 & 0.48 & 0.32 \\
    \shortstack[l]{\textcolor{gray}{Avg. lottery-win-year MPC}                                                                                                         \\ \textcolor{gray}{(model, splurge=0)}} & \textcolor{gray}{0.69} & \textcolor{gray}{0.53} & \textcolor{gray}{0.36} & \textcolor{gray}{0.14}
    \\ \hline
  \end{tabular}
\end{frame}

\begin{frame}
  \frametitle{Untargeted moments (2)}
  \begin{columns}
    \begin{column}{0.50\textwidth}
      \begin{figure}
        \includegraphics[width=.9\linewidth]{\PathToRoot/images/IMPCs_wSplEstimated}
        \caption{Share of lottery win spent} 
      \end{figure}
    \end{column}
    \begin{column}{0.50\textwidth}
      \begin{figure}
        \includegraphics[width=.9\linewidth]{\PathToRoot/Code/HA-Models/FromPandemicCode/Figures/UnempSpell_Dynamics}
        \caption{Spending upon expiry of UI benefits} 
      \end{figure}
    \end{column}
  \end{columns}
  \begin{itemize}
    \item
          Ganong and Noel (2019): UI expiry $\Rightarrow$ drop of 12 percent (month)
    \item
          Our model $\Rightarrow$ drop of 18 percent (quarter)
  \end{itemize}
\end{frame}

\begin{frame}
  \frametitle{Multipliers}

  \begin{columns}

    \begin{column}{0.50\textwidth}
      \begin{equation*}
        M^P_t = \frac{\text{NPV of induced consumption up to $t$}}{\text{NPV of the cost of the policy}}
      \end{equation*}
    \end{column}

    \begin{column}{0.50\textwidth}

      \begin{figure}[t]
        \centering
        \includegraphics[width=\linewidth]{Code/HA-Models/FromPandemicCode/Figures/Cumulative_multipliers}
      \end{figure}

    \end{column}
  \end{columns}

  \begin{table}[t]
    \begin{tabular}
      {@{}lccc@{}}
      \hline
                                                   & Stimulus check & UI extension & Tax cut \\  \hline
      10y-horizon Multiplier (no AD effect)        & 0.88           & 0.91         & 0.85    \\
      10y-horizon Multiplier (AD effect)           & 1.23           & 1.21         & 0.98    \\
      % 10y-horizon (1st round AD effect only) &1.157  & 1.148  & 0.951     \\ 
      Share of policy expenditure during recession & 100.0\%        & 79.6\%       & 57.8 \% \\
    \end{tabular}
  \end{table}
\end{frame}

\begin{frame}
  \frametitle{Robustness: Multipliers in a HANK and SAM model --- Setup}
  \begin{itemize}
    \itemsep = .5\bigskipamount
    \item
          Evaluate the policies in a relatively standard HANK and SAM model (Du, 2024)
    \item
          New Keynesian: Monopolistic competition + sticky prices
    \item
          Search and matching: Random search, labor market tightness affects job finding and vacancy filling probabilities
    \item
          Government policy: Monetary and fiscal rules
    \item
          Fiscal multipliers through an intertemporal Keynesian cross mechanism \\[1ex]
          However: No state dependence
    \item
          Solution method $\Rightarrow$ cannot evaluate effects starting in a deep recessionary state \\[1ex]
          This also implies that we cannot use our welfare measure
  \end{itemize}
\end{frame}

\begin{frame}
  \frametitle{Robustness: Multipliers in a HANK and SAM model --- Results}
  \begin{figure}
    \begin{center}
      \includegraphics[scale=0.6]{Code/HA-Models/FromPandemicCode/Figures/Cumulative_multipliers_withHank}
    \end{center}
    \vspace{0.2cm}
    \captionof{figure}{HA w/AD effects + HANK and SAM} 
    %\hfill
    %\begin{minipage}[c]{0.48\linewidth}
    %\includegraphics[scale= 0.5]{Code/HA-Models/FromPandemicCode/Figures/Cumulative_multipliers_HANK}
    %\vspace{0.2cm}
    %\captionof{figure}{HANK XXupdate}
    %\end{minipage}
  \end{figure}
\end{frame}

\begin{frame}
  \frametitle{Welfare measure}
  \begin{itemize}[<+->]
    \item
          Aim: Welfare measure does not reflect benefits of redistribution in ``normal'' times
    \item
          Want: Utility-based measure of benefits of implementing a policy in a recession
    \item
          Welfare weights: $u'(\mathbf{c}_{it,\textit{normal}})$
    \item
          Measure for a given $policy$ with $Rec,AD\in\{0,1\}$
  \end{itemize}

  \begin{equation*}
    \mathcal{W}(\text{policy},Rec,AD) =\frac{1}{\mathcal{N}} \sum_{i=1}^{N} \sum_{t=0}^{\infty} \frac{1}{R^t} \frac{u(\mathbf{c}_{it,\textit{policy},Rec,AD}) - u(\mathbf{c}_{it,\textit{none},Rec,AD})}{ u'(\mathbf{c}_{it,\textit{normal}})}
  \end{equation*}
  where $\mathcal{N} = NPV(\text{policy},Rec,AD)$

  \begin{itemize}[<+->]
    \item
          Normal times: $\mathcal{W}(\text{policy},0,0) = 1$ (for $\Delta \mathbf{c}_{it}\approx 0$)
  \end{itemize}
\end{frame}

\begin{frame}
  \frametitle{Welfare results}
  \centering
  \begin{tabular}
    {@{}lccc@{}}
    \hline
                                              & Stimulus check & UI extension & Tax cut \\  \hline
    $\mathcal{W}(\text{policy}, Rec=0, AD=0)$ & 0.96           & 0.85         & 0.99    \\
    $\mathcal{W}(\text{policy}, Rec=1, AD=0)$ & 1.00           & 1.83         & 0.97    \\
    $\mathcal{W}(\text{policy}, Rec=1, AD=1)$ & 1.35           & 2.15         & 1.11    \\ \hline
  \end{tabular}
  \medskip
  \begin{itemize}[<+->]
    \itemsep = .75\bigskipamount
    \item
          Normal times: Welfare of UI extension $< 1$ due to concavity of $u(\cdot)$ \\[1ex]
          Relatively large change in $\mathbf{c}_{it}$ for small number of households
    \item
          $AD=0$: Benefit of UI extension since recession increases unemployment $\Rightarrow$ increased marginal utility for affected households
    \item
          $AD=1$: Stimulating spending during recession increases measure for all policies
  \end{itemize}
\end{frame}

\begin{frame}
  \frametitle{Conclusion: Comparing the policies}
  \begin{itemize}[<+->]
    \itemsep = .5\bigskipamount
    \item
          Comparison of three consumption stimulus policies in a HA model consistent with data on the distribution of liquid wealth and intertemporal MPCs
    \item
          Welfare measure: UI extension is the clear bang-for-the-buck winner
    \item
          The stimulus check is less well targeted, but\ldots
          \begin{itemize}[<+->]
            \itemsep = .25\bigskipamount
            \item
                  is transferred immediately ensuring that money arrives when it is most valuable
            \item
                  is more easily scaled up to provide more stimulus
          \end{itemize}
    \item
          The tax cut is both poorly targeted and may yield substantial spending after the recession is over
    \item
          Framework can be used to evaluate other candidate policies

  \end{itemize}

\end{frame}

\begin{frame}
  \frametitle{Thank you for your attention!}
  \begin{itemize}[<+->]
    \item
          Access the paper, presentation slides and code at: \href{https://github.com/econ-ark/HAFiscal}{https://github.com/econ-ark/HAFiscal}
  \end{itemize}

  \begin{figure}
    \centering
    \includegraphics[width=0.3\linewidth]{@local/HAFiscal-Slides-qr-code}
  \end{figure}

\end{frame}

\section{Appendix}

\begin{frame}
  \frametitle{Parameters --- same for all types}
  \label{sli:paramsSame}
  %\begin{tabular}{lcd{3}} 
  \begin{tabular}{lcc}
    \hline
    %	\multicolumn{3}{l}{Parameters that apply to all types} \\ \hline	
    Parameter                                                   & Notation                     & \text{Value}                          \\ \hline
    Risk aversion                                               & $\gamma$                     & 2.0                                   \\
    Splurge                                                     & $\varsigma$                  & 0.249                                 \\
    Survival probability, quarterly                             & $1-D$                        & 0.994                                 \\
    Risk free interest rate, quarterly (gross)                  & $R$                          & 1.01                                  \\
    Standard deviation of transitory shock                      & $\sigma_\xi$                 & 0.346                                 \\
    Standard deviation of permanent shock                       & $\sigma_\psi$                & 0.0548                                \\
    Unemployment benefits replacement rate (share of PI)        & \textcolor{red}{$\rho_b$}    & \textcolor{red}{0}.\textcolor{red}{7} \\
    Unemployment income w/o benefits (share of PI)              & \textcolor{red}{$\rho_{nb}$} & \textcolor{red}{0}.\textcolor{red}{5} \\
    Avg. duration of unemp. benefits in normal times (quarters) &                              & 2                                     \\
    Avg. duration of unemp. spell in normal times (quarters)    &                              & 1.5                                   \\
    %	Probability of leaving unemployment & $\pi_{ue}$ & 0.667 \\ 
    Consumption elasticity of aggregate demand effect           & $\kappa$                     & 0.3
    \\ \hline
  \end{tabular}

  \vspace{.5cm}
  \hyperlink{Parameters}{\beamerbutton{Go back}}
\end{frame}

\begin{frame}
  \frametitle{Parameters describing the policies}
  \label{sli:policies}
  \begin{center}
    \begin{tabular}{lc}
      \hline
      \multicolumn{2}{l}{Parameters describing policy experiments} \\ \hline
      Parameter                                     & Value        \\ \hline
      Change in unemployment rates in a recession   & $\times 2$   \\
      Expected unemployment spell in a recession    & 4 quarters   \\
      Average length of recession                   & 6 quarters   \\
      Size of stimulus check                        & \$1,200      \\
      PI threshold for reducing check size          & \$100,000    \\
      PI threshold for not receiving check          & \$150,000    \\
      Extended unemployment benefits                & 4 quarters   \\
      Length of payroll tax cut                     & 8 quarters   \\
      Income increase from payroll tax cut          & 2 percent    \\
      Belief (probability) that tax cut is extended & 50 percent
      \\ \hline
    \end{tabular}
  \end{center}
  \vspace{.5cm}
  \hyperlink{Parameters}{\beamerbutton{Go back}}
\end{frame}

\begin{frame}
  \frametitle{Robustness: Model w/o splurge consumption}
  \begin{columns}
    \begin{column}{0.5\textwidth}
      \includegraphics[width=\linewidth]{\PathToRoot/images/IMPCs_both}
    \end{column}
    \begin{column}{0.5\textwidth}
      \includegraphics[width=\linewidth]{Code/HA-Models/FromPandemicCode/Figures/Splurge0/Cumulative_multipliers_SplurgeComp}
    \end{column}
  \end{columns}
  \begin{center}
    \begin{tabular}
      {@{}lccc@{}}
      \hline
                                                & Stimulus check & UI extension & Tax cut    \\  \hline
      %	$\mathcal{W}(\text{policy}, Rec=0, AD=0)$ & 0.97(0.96)  & 0.84(0.85)  & 0.99(0.99)     \\ 
      %	$\mathcal{W}(\text{policy}, Rec=1, AD=0)$ & 1.00(1.00)  & 1.80(1.83)  & 0.97(0.97)     \\ 
      $\mathcal{W}(\text{policy}, Rec=1, AD=1)$ & 1.27(1.35)     & 2.12(2.15)   & 1.09(1.11) \\ \hline
    \end{tabular}
  \end{center}
\end{frame}

\end{document}

% Booktabs commands used: \toprule (6), \midrule (22), \bottomrule (6), \cmidrule (0), \addlinespace (0)
